% This text is proprietary.
% It's a part of presentation made by myself.
% It may not used commercial.
% The noncommercial use such as private and study is free
% Sep. 2005
% Author: Sascha Frank
% University Freiburg
% www.informatik.uni-freiburg.de/~frank/


\documentclass{beamer}
\usepackage{graphicx}

%%%%%%%%%%%%%%%%%%%%%%%%%%%%MACROS

\def\var#1{\mbox{\textrm{\bf{\sc{#1}}}}}
\newcommand{\set}[1]{\{#1\}}
\newcommand{\V}{\mathbb{V}}
\newcommand{\E}{\mathbb{E}}
\newcommand{\size}[1]{\ensuremath{\left|#1\right|}}

\newcommand {\framedgraphic}[3] {
    \frame{\frametitle{#1}
        \begin{center}
            \includegraphics[width=\textwidth,height=#3\textheight,keepaspectratio]{#2}
        \end{center}
    }
}
\newcommand{\disableframe}[1]{}

\def\func#1#2{\mbox{\textrm{\bf{\sc{#1}}}}\ensuremath{(~{#2}~)}}
% pseudocode notations
\newcommand{\T}{\mbox{\hspace{0.5cm}}}

% asymptotic notations
\newcommand{\Oh}[1]{\ensuremath{{\mathcal O}\left({#1}\right)}}
\newcommand{\Om}[1]{\ensuremath{{\Omega}\left({#1}\right)}}
\newcommand{\Th}[1]{\ensuremath{{\Theta}\left({#1}\right)}}

\newcommand{\ceil}[1]{\ensuremath{\left\lceil#1\right\rceil}}
\newcommand{\floor}[1]{\ensuremath{\left\lfloor#1\right\rfloor}}
\newcommand{\maybepause}{
    \iffalse
    \pause
    \fi
}

%%%%%%%%%%%%%%%%%%%%%%%%%%%%MACROS

\begin{document}
\title{Clean Breadth First Search}
\author{Yonatan R. Fogel}
\date{\today}

\frame{\titlepage}

%TODO: Maybe re-add table of contents
%\frame{\frametitle{Table of contents}\tableofcontents}


\section{Introduction}
\frame{\frametitle{BFS Problem Description}
    \begin{itemize}
        \item Given
            \begin{itemize}
                \item Graph $G=(\V,\E)$
                \item Source vertex $s \in \V$
            \end{itemize}
        \item Calculate
            \begin{itemize}
                \item $Dist_{u\in \V} =$ length of the shortest path from $s$ to $u$ in $G$
                \item $Parent_{u\in \V} = v \in \V$ s.t. $(v, u) \in \E, Dist_u = Dist_v + 1$
            \end{itemize}
    \end{itemize}
}

\framedgraphic{BFS Example}{bfs-1.png}{1.0}

\frame{\frametitle{Terminology}
    \begin{itemize}
        \item $n= \size{\V}$ the number of nodes in a graph \maybepause
        \item $m= \size{\E}$ the number of edges in a graph \maybepause
        \item $D= \Oh{n}$ the diameter of a graph \maybepause
        \item $\Gamma_u =$ the set of vertexes adjacent to $u$ \maybepause
        \item $T_s =$ the running time for a serial algorithm \maybepause
        \item $T_P =$ the running time for a parallel algorithm running on $P$ cores \maybepause
        \item $T_\infty =$ the running time for a parallel algorithm running on infinite cores
        \item $T_1 = \Om{T_s} =$ the running time for a parallel algorithm running on one core \maybepause
        \item $W_P = \Om{T_s} =$ the total work done by a parallel algorithm running on $P$ cores (excluding idle time)
            \begin{itemize}
                \item Reducing $W_P$ can reduce energy use \cite{race-to-idle}
            \end{itemize}
            %TODO: motivation for W_P
    \end{itemize}
}

\frame{\frametitle{Serial-BFS}
    \begin{enumerate}
        \item for each vertex $u \in \V$
        \item \T $Dist_u \gets \infty$
        \item $Dist_s \gets 0$
        \item $Q \gets \emptyset$
        \item $\func{Enqueue}{Q, ~s}$
        \item while $Q \neq \emptyset$ do
        \item \T $u \gets \func{Dequeue}{Q}$
        \item \T for each vertex $v$ in $\Gamma(u)$ do
        \item \T \T if $Dist_v = \infty$ then
        \item \T \T \T $Dist_v \gets Dist_u + 1$
        \item \T \T \T $\func{Enqueue}{Q, v}$
    \end{enumerate}
}

\frame{\frametitle{Computation Model}
    \begin{itemize}
        \item Large shared memory
        \item Consistent caches between cores
        \item Synchronizing $y$ tasks takes $T_\infty=\Th{\log y}$ time
        \item Cilk has this model with randomized work stealing \cite{cilk}
    \end{itemize}
}

\frame{\frametitle{Motivation for (P)BFS}
    BFS is used for
    \begin{itemize}
        \item Path Finding 
            \begin{itemize}
                \item Video Games
                \item Google Maps\maybepause
            \end{itemize}
        \item Analyzing social networks \maybepause %TODO: ref
        \item Designing and analyzing VLSI \maybepause %TODO: ref
        \item Task scheduling \maybepause %TODO: ref
        \item As a primitive in other algorithms
    \end{itemize}
}

\disableframe{\frametitle{Motivation for PBFS}
    BFS can achieve significant speedup for practical graphs
    %TODO: maybe delete this slide?  Or add more
}

\frame{\frametitle{Existing Approaches for PBFS}
    \begin{itemize}
        \item Assumes PRAM model \maybepause
        \item Specialized for specific hardware \cite{blue-bfs}
            \begin{itemize}
                \item GPU
                \item CRAY (hardware mutex every $64$ bits, atomic add) \maybepause
                    \cite{cray-bfs}
            \end{itemize}
        \item Uses atomic instructions \cite{block-queue-bfs}
        \item Specialized for sparse (or dense) graphs only
        \item Specialized for bounded out-degree (not scale-free) \cite{mit-bag} \maybepause
        \item $T_1$ or $T_p$ is not asymptotically optimal \maybepause
        \item Room for energy efficiency improvements (non-optimal $W_P$) \maybepause
        \item Offloads some work to scheduler
            \begin{itemize}
                \item Work-stealing (randomized) gives high probability bounds
                    \cite{mit-bag}
            \end{itemize}
        \item Non level-synchronous \cite{distinguished-bfs}
    \end{itemize}
}

\frame{\frametitle{Existing Algorithms}
    \begin{itemize}
        \item MIT-bag \cite{mit-bag} uses penants, bags, and reducer hyperobjects \pause
        \item block-queue-bfs \cite{block-queue-bfs} allocates space from FIFO in blocks \pause
        \item distinguished-bfs \cite{distinguished-bfs} contracts the graph to make it dense \pause
        \item cray-bfs \cite{cray-bfs} uses fast hardware mutexes and atomic increments
    \end{itemize}
}

\frame{\frametitle{Level-Synchronous BFS}
    \begin{itemize}
        \item All nodes at distance $d$ from $s$ are processed before any nodes at distance $d^\prime > d$. \maybepause
        \item $T_P = \Om{n/p+m/p+D\log P}$ \maybepause
            \begin{itemize}
                \item $T_p = \Om{T_s/P} = \Om{n/p+m/p}$ \maybepause
                \item Let $n_\ell, m_\ell$ be the number of nodes and edges visited at level $\ell$. \maybepause
                \item Consider a graph where $\forall_{0<\ell\leq D}n_\ell+m_\ell=\Th{P}$
                \item Every level has $\Th{P}$ work and uses $P_\ell \leq P$ cores
                \item For each level, $T_p = \Om{P/P_\ell + \log P_\ell} = \Om{\log P}$ time
            \end{itemize}
            %TODO: improve description of problem
    \end{itemize}
}

\disableframe{\frametitle{Bottlenecks for Parallelizing BFS}
    \begin{itemize}
        \item FIFO
        \item $Dist$ array
    \end{itemize}
}

\section{Clean BFS}
\frame{\frametitle{Clean BFS - Properties}
    \begin{itemize}
        \item Recall $T_s = \Oh{n+m}$
        \item $T_1 = \Oh{n+m}$ (optimal)
        \item $W_P = \Oh{n+m}$ (optimal)
        \item $T_P = \Oh{n/p+m/p+D\log P}$ (optimal)
        \item Scale-free
        \item Deterministic worst case bounds
    \end{itemize}
}

\frame{\frametitle{Clean BFS - High Level}
    For each level $\ell \in [0\ldots D)$
    \begin{enumerate}
        \item To get optimal $T_1,T_P$
        \begin{enumerate}
            \item Prepare to Split Work
            \item Split Work and Process Edges
            \item Dedup Vertexes and Combine Queues
        \end{enumerate}
    \item To get optimal $W_P$
        \begin{enumerate}
            \item Reduce Search Space
            \item Dynamically Choose Number of Cores
        \end{enumerate}
    \end{enumerate}
}

\subsection{Algorithm}
\frame{\frametitle{Clean BFS - Prepare to Split Work}
    \begin{itemize}
        \item One input queue $Q_{in}\subseteq \V$.
            \begin{itemize}
                \item Each vertex in $Q_{in}$ is unique
                \item $\forall u (u \in Q_{in} \Rightarrow Dist_u = \ell)$
                \item $\forall u (u \in Q_{in} \Rightarrow \size{\Gamma_u} > 0$
            \end{itemize}
        \item Generate $OutDegrees[0\leq i<\size{Q_{in}}] = \size{\Gamma(Q_{in}[i])}$ in parallel
        \item Perform a parallel prefix sum on $OutDegrees$
            \begin{tabular}{r|l|l|l|l|l|l|}
                \cline{2-7}
                $Q_{in}=$ &1 & 3 & 2 & 4 & $\ldots$ & $\ldots$ \\ \cline{2-7}
  $OutDegrees_{before} =$ &1 & 3 & 2 & 4 & $\ldots$ & $\ldots$ \\ \cline{2-7}
  $OutDegrees_{after}  =$ &1 & 4 & 6 & 10 & $\ldots$ & $m_\ell$ \\ \cline{2-7}
            \end{tabular}
    \end{itemize}
}

\frame{\frametitle{Clean BFS - Split Work and Process Edges}
    \begin{itemize}
        \item Each core $i$ processes $m_\ell/P$ edges
            \begin{itemize}
                \item searches $OutDegrees$ for $1+\floor{\frac{i ~ m_\ell}{P}}$ to find starting edge
                \item does $\Oh{\log \frac{n_\ell}{P} + \log P}$ work
                \item processes $\floor{\frac{m_\ell}{P}}$ consecutive edges
                    \begin{itemize}
                        \item $Q_i \gets \emptyset$
                        \item for each edge $(u,v)$
                        \item \T if $Dist_v = \infty$ then
                        \item \T \T $Dist_v \gets Dist_u + 1$
                        \item \T \T $Owner_v \gets i$
                        \item \T \T $\func{Enqueue}{Q_i, v}$
                    \end{itemize}
                \item Benign race conditions
            \end{itemize}
    \end{itemize}
}

\frame{\frametitle{Clean BFS - Dedup Vertexes and Combine Queues}
    \begin{itemize}
        \item $Size_{-1} = 0$
        \item Each core $i$ uses $Owner$ to ensure each vertex lives in at most one output queue
            \begin{itemize}
                \item $Q_i \gets \set{u \in Q_i : Owner_u = i}$
                \item $Size_i \gets \size{Q_i}$
            \end{itemize}
        \item Perform a parallel prefix sum on $Size$
        \item Each core $i$ copies its queue back into $Q_{in}$ at offset $Size_{i-1}$
    \end{itemize}
}

\frame{\frametitle{Clean BFS - Reduce Search Space}
    \begin{itemize}
        \item $N \gets \size{OutDegrees}$
        \item Each core $i$
            \begin{itemize}
                \item $FirstDegree \gets OutDegrees[\floor{\frac{iN}{P}}-1]$
                \item $FirstDegreeNext \gets OutDegrees[\floor{\frac{(i+1)N}{P}}-1]$
                \item $FirstCore \gets \ceil{\frac{P~ FirstDegree}{m_\ell}}$
                \item $LastCore \gets \ceil{\frac{P~ FirstDegreeNext}{m_\ell}}$
                \item parallel for $j \gets FirstCore$ to $LastCore$
                \item \T $SubList_j \gets i$
            \end{itemize}
        \item Using $SubList_i$, core $i$ can search only $n_\ell/p$ indexes
        \item $W_P$ goes from $\Oh{n+m+DP\log P}$ to $\Oh{n+m+DP}$
    \end{itemize}
}

\frame{\frametitle{Clean BFS - Dynamically Choose Number of Cores}
    \begin{itemize}
        \item Immediately after calculating $m_\ell$
        \item $P_\ell \gets \min(m_\ell, P)$
        \item Use at most $P_\ell$ cores until next time $m_\ell$ is calculated.
        \item This ensures the $\Oh{P_\ell}$ work every level is $\Oh{m_\ell}$ and can be absorbed into the constant.
        \item $W_P$ goes from $\Oh{n+m+DP}$ to $\Oh{n+m+D}=\Oh{n+m}$
    \end{itemize}
}

\frame{\frametitle{Future Work}
    \begin{itemize}
        \item Optimize Clean BFS for the PRAM model
            \begin{itemize}
                \item Clean BFS runs in same time for PRAM but is not asymptotically optimal
            \end{itemize}
        \item Modify Clean BFS to remove false sharing
    \end{itemize}
}

\iffalse
        \item Combined over all levels
            \begin{itemize}
                \item $T_P = \Oh{n/P + m/P + D\log P}$
                \item $T_1 = \Oh{n+m}$
                \item $W_P = \Oh{n+m+DP\log P}$\maybepause
            \end{itemize}
        \item 
\fi



\begin{frame}[allowframebreaks]
    \frametitle{References}
    %\bibliographystyle{amsalpha}
    \bibliographystyle{apalike}
    \bibliography{../source/bib/paper.bib}
\end{frame}

\end{document}






%TODO:
Add figures/slides/references for:


False sharing
CILK+

Related work
    m




































\iffalse
%\subsection{Subsection no.1.1  }
%\frame{
%Without title somethink is missing.
%}


\section{Section no. 2}
\subsection{Lists I}
\frame{\frametitle{unnumbered lists}
\begin{itemize}
\item Introduction to  \LaTeX
\item Course 2
\item Termpapers and presentations with \LaTeX
\item Beamer class
\end{itemize}
}

\frame{\frametitle{lists with maybepause}
\begin{itemize}
\item Introduction to  \LaTeX \maybepause
\item Course 2 \maybepause
\item Termpapers and presentations with \LaTeX \maybepause
\item Beamer class
\end{itemize}
}

\subsection{Lists II}
\frame{\frametitle{numbered lists}
\begin{enumerate}
\item Introduction to  \LaTeX
\item Course 2
\item Termpapers and presentations with \LaTeX
\item Beamer class
\end{enumerate}
}
\frame{\frametitle{numbered lists with maybepause}
\begin{enumerate}
\item Introduction to  \LaTeX \maybepause
\item Course 2 \maybepause
\item Termpapers and presentations with \LaTeX \maybepause
\item Beamer class
\end{enumerate}
}

\section{Section no.3}
\subsection{Tables}
\frame{\frametitle{Tables}
\begin{tabular}{|c|c|c|}
\hline
\textbf{Date} & \textbf{Instructor} & \textbf{Title} \\
\hline
WS 04/05 & Sascha Frank & First steps with  \LaTeX  \\
\hline
SS 05 & Sascha Frank & \LaTeX \ Course serial \\
\hline
\end{tabular}}


\frame{\frametitle{Tables with maybepause}
\begin{tabular}{c c c}
A & B & C \\
\maybepause
1 & 2 & 3 \\
\maybepause
A & B & C \\
\end{tabular} }


\section{Section no. 4}
\subsection{blocs}
\frame{\frametitle{blocs}

\begin{block}{title of the bloc}
bloc text
\end{block}

\begin{exampleblock}{title of the bloc}
bloc text
\end{exampleblock}


\begin{alertblock}{title of the bloc}
bloc text
\end{alertblock}
}
\fi
