
%% bare_conf.tex
%% V1.3
%% 2007/01/11
%% by Michael Shell
%% See:
%% http://www.michaelshell.org/
%% for current contact information.
%%
%% This is a skeleton file demonstrating the use of IEEEtran.cls
%% (requires IEEEtran.cls version 1.7 or later) with an IEEE conference paper.
%%
%% Support sites:
%% http://www.michaelshell.org/tex/ieeetran/
%% http://www.ctan.org/tex-archive/macros/latex/contrib/IEEEtran/
%% and
%% http://www.ieee.org/

%%*************************************************************************
%% Legal Notice:
%% This code is offered as-is without any warranty either expressed or
%% implied; without even the implied warranty of MERCHANTABILITY or
%% FITNESS FOR A PARTICULAR PURPOSE! 
%% User assumes all risk.
%% In no event shall IEEE or any contributor to this code be liable for
%% any damages or losses, including, but not limited to, incidental,
%% consequential, or any other damages, resulting from the use or misuse
%% of any information contained here.
%%
%% All comments are the opinions of their respective authors and are not
%% necessarily endorsed by the IEEE.
%%
%% This work is distributed under the LaTeX Project Public License (LPPL)
%% ( http://www.latex-project.org/ ) version 1.3, and may be freely used,
%% distributed and modified. A copy of the LPPL, version 1.3, is included
%% in the base LaTeX documentation of all distributions of LaTeX released
%% 2003/12/01 or later.
%% Retain all contribution notices and credits.
%% ** Modified files should be clearly indicated as such, including  **
%% ** renaming them and changing author support contact information. **
%%
%% File list of work: IEEEtran.cls, IEEEtran_HOWTO.pdf, bare_adv.tex,
%%                    bare_conf.tex, bare_jrnl.tex, bare_jrnl_compsoc.tex
%%*************************************************************************

% *** Authors should verify (and, if needed, correct) their LaTeX system  ***
% *** with the testflow diagnostic prior to trusting their LaTeX platform ***
% *** with production work. IEEE's font choices can trigger bugs that do  ***
% *** not appear when using other class files.                            ***
% The testflow support page is at:
% http://www.michaelshell.org/tex/testflow/



% Note that the a4paper option is mainly intended so that authors in
% countries using A4 can easily print to A4 and see how their papers will
% look in print - the typesetting of the document will not typically be
% affected with changes in paper size (but the bottom and side margins will).
% Use the testflow package mentioned above to verify correct handling of
% both paper sizes by the user's LaTeX system.
%
% Also note that the "draftcls" or "draftclsnofoot", not "draft", option
% should be used if it is desired that the figures are to be displayed in
% draft mode.
%
\documentclass[conference]{IEEEtran}
% Add the compsoc option for Computer Society conferences.
%
% If IEEEtran.cls has not been installed into the LaTeX system files,
% manually specify the path to it like:
% \documentclass[conference]{../sty/IEEEtran}





% Some very useful LaTeX packages include:
% (uncomment the ones you want to load)


% *** MISC UTILITY PACKAGES ***
%
%\usepackage{ifpdf}
% Heiko Oberdiek's ifpdf.sty is very useful if you need conditional
% compilation based on whether the output is pdf or dvi.
% usage:
% \ifpdf
%   % pdf code
% \else
%   % dvi code
% \fi
% The latest version of ifpdf.sty can be obtained from:
% http://www.ctan.org/tex-archive/macros/latex/contrib/oberdiek/
% Also, note that IEEEtran.cls V1.7 and later provides a builtin
% \ifCLASSINFOpdf conditional that works the same way.
% When switching from latex to pdflatex and vice-versa, the compiler may
% have to be run twice to clear warning/error messages.






% *** CITATION PACKAGES ***
%
\usepackage{cite}
% cite.sty was written by Donald Arseneau
% V1.6 and later of IEEEtran pre-defines the format of the cite.sty package
% \cite{} output to follow that of IEEE. Loading the cite package will
% result in citation numbers being automatically sorted and properly
% "compressed/ranged". e.g., [1], [9], [2], [7], [5], [6] without using
% cite.sty will become [1], [2], [5]--[7], [9] using cite.sty. cite.sty's
% \cite will automatically add leading space, if needed. Use cite.sty's
% noadjust option (cite.sty V3.8 and later) if you want to turn this off.
% cite.sty is already installed on most LaTeX systems. Be sure and use
% version 4.0 (2003-05-27) and later if using hyperref.sty. cite.sty does
% not currently provide for hyperlinked citations.
% The latest version can be obtained at:
% http://www.ctan.org/tex-archive/macros/latex/contrib/cite/
% The documentation is contained in the cite.sty file itself.






% *** GRAPHICS RELATED PACKAGES ***
%
\ifCLASSINFOpdf
  % \usepackage[pdftex]{graphicx}
  % declare the path(s) where your graphic files are
  % \graphicspath{{../pdf/}{../jpeg/}}
  % and their extensions so you won't have to specify these with
  % every instance of \includegraphics
  % \DeclareGraphicsExtensions{.pdf,.jpeg,.png}
\else
  % or other class option (dvipsone, dvipdf, if not using dvips). graphicx
  % will default to the driver specified in the system graphics.cfg if no
  % driver is specified.
  % \usepackage[dvips]{graphicx}
  % declare the path(s) where your graphic files are
  % \graphicspath{{../eps/}}
  % and their extensions so you won't have to specify these with
  % every instance of \includegraphics
  % \DeclareGraphicsExtensions{.eps}
\fi
% graphicx was written by David Carlisle and Sebastian Rahtz. It is
% required if you want graphics, photos, etc. graphicx.sty is already
% installed on most LaTeX systems. The latest version and documentation can
% be obtained at: 
% http://www.ctan.org/tex-archive/macros/latex/required/graphics/
% Another good source of documentation is "Using Imported Graphics in
% LaTeX2e" by Keith Reckdahl which can be found as epslatex.ps or
% epslatex.pdf at: http://www.ctan.org/tex-archive/info/
%
% latex, and pdflatex in dvi mode, support graphics in encapsulated
% postscript (.eps) format. pdflatex in pdf mode supports graphics
% in .pdf, .jpeg, .png and .mps (metapost) formats. Users should ensure
% that all non-photo figures use a vector format (.eps, .pdf, .mps) and
% not a bitmapped formats (.jpeg, .png). IEEE frowns on bitmapped formats
% which can result in "jaggedy"/blurry rendering of lines and letters as
% well as large increases in file sizes.
%
% You can find documentation about the pdfTeX application at:
% http://www.tug.org/applications/pdftex





% *** MATH PACKAGES ***
%
%\usepackage[cmex10]{amsmath}
% A popular package from the American Mathematical Society that provides
% many useful and powerful commands for dealing with mathematics. If using
% it, be sure to load this package with the cmex10 option to ensure that
% only type 1 fonts will utilized at all point sizes. Without this option,
% it is possible that some math symbols, particularly those within
% footnotes, will be rendered in bitmap form which will result in a
% document that can not be IEEE Xplore compliant!
%
% Also, note that the amsmath package sets \interdisplaylinepenalty to 10000
% thus preventing page breaks from occurring within multiline equations. Use:
%\interdisplaylinepenalty=2500
% after loading amsmath to restore such page breaks as IEEEtran.cls normally
% does. amsmath.sty is already installed on most LaTeX systems. The latest
% version and documentation can be obtained at:
% http://www.ctan.org/tex-archive/macros/latex/required/amslatex/math/





% *** SPECIALIZED LIST PACKAGES ***
%
%\usepackage{algorithmic}
% algorithmic.sty was written by Peter Williams and Rogerio Brito.
% This package provides an algorithmic environment fo describing algorithms.
% You can use the algorithmic environment in-text or within a figure
% environment to provide for a floating algorithm. Do NOT use the algorithm
% floating environment provided by algorithm.sty (by the same authors) or
% algorithm2e.sty (by Christophe Fiorio) as IEEE does not use dedicated
% algorithm float types and packages that provide these will not provide
% correct IEEE style captions. The latest version and documentation of
% algorithmic.sty can be obtained at:
% http://www.ctan.org/tex-archive/macros/latex/contrib/algorithms/
% There is also a support site at:
% http://algorithms.berlios.de/index.html
% Also of interest may be the (relatively newer and more customizable)
% algorithmicx.sty package by Szasz Janos:
% http://www.ctan.org/tex-archive/macros/latex/contrib/algorithmicx/




% *** ALIGNMENT PACKAGES ***
%
%\usepackage{array}
% Frank Mittelbach's and David Carlisle's array.sty patches and improves
% the standard LaTeX2e array and tabular environments to provide better
% appearance and additional user controls. As the default LaTeX2e table
% generation code is lacking to the point of almost being broken with
% respect to the quality of the end results, all users are strongly
% advised to use an enhanced (at the very least that provided by array.sty)
% set of table tools. array.sty is already installed on most systems. The
% latest version and documentation can be obtained at:
% http://www.ctan.org/tex-archive/macros/latex/required/tools/


%\usepackage{mdwmath}
%\usepackage{mdwtab}
% Also highly recommended is Mark Wooding's extremely powerful MDW tools,
% especially mdwmath.sty and mdwtab.sty which are used to format equations
% and tables, respectively. The MDWtools set is already installed on most
% LaTeX systems. The lastest version and documentation is available at:
% http://www.ctan.org/tex-archive/macros/latex/contrib/mdwtools/


% IEEEtran contains the IEEEeqnarray family of commands that can be used to
% generate multiline equations as well as matrices, tables, etc., of high
% quality.


%\usepackage{eqparbox}
% Also of notable interest is Scott Pakin's eqparbox package for creating
% (automatically sized) equal width boxes - aka "natural width parboxes".
% Available at:
% http://www.ctan.org/tex-archive/macros/latex/contrib/eqparbox/





% *** SUBFIGURE PACKAGES ***
%\usepackage[tight,footnotesize]{subfigure}
% subfigure.sty was written by Steven Douglas Cochran. This package makes it
% easy to put subfigures in your figures. e.g., "Figure 1a and 1b". For IEEE
% work, it is a good idea to load it with the tight package option to reduce
% the amount of white space around the subfigures. subfigure.sty is already
% installed on most LaTeX systems. The latest version and documentation can
% be obtained at:
% http://www.ctan.org/tex-archive/obsolete/macros/latex/contrib/subfigure/
% subfigure.sty has been superceeded by subfig.sty.



%\usepackage[caption=false]{caption}
%\usepackage[font=footnotesize]{subfig}
% subfig.sty, also written by Steven Douglas Cochran, is the modern
% replacement for subfigure.sty. However, subfig.sty requires and
% automatically loads Axel Sommerfeldt's caption.sty which will override
% IEEEtran.cls handling of captions and this will result in nonIEEE style
% figure/table captions. To prevent this problem, be sure and preload
% caption.sty with its "caption=false" package option. This is will preserve
% IEEEtran.cls handing of captions. Version 1.3 (2005/06/28) and later 
% (recommended due to many improvements over 1.2) of subfig.sty supports
% the caption=false option directly:
%\usepackage[caption=false,font=footnotesize]{subfig}
%
% The latest version and documentation can be obtained at:
% http://www.ctan.org/tex-archive/macros/latex/contrib/subfig/
% The latest version and documentation of caption.sty can be obtained at:
% http://www.ctan.org/tex-archive/macros/latex/contrib/caption/




% *** FLOAT PACKAGES ***
%
%\usepackage{fixltx2e}
% fixltx2e, the successor to the earlier fix2col.sty, was written by
% Frank Mittelbach and David Carlisle. This package corrects a few problems
% in the LaTeX2e kernel, the most notable of which is that in current
% LaTeX2e releases, the ordering of single and double column floats is not
% guaranteed to be preserved. Thus, an unpatched LaTeX2e can allow a
% single column figure to be placed prior to an earlier double column
% figure. The latest version and documentation can be found at:
% http://www.ctan.org/tex-archive/macros/latex/base/



%\usepackage{stfloats}
% stfloats.sty was written by Sigitas Tolusis. This package gives LaTeX2e
% the ability to do double column floats at the bottom of the page as well
% as the top. (e.g., "\begin{figure*}[!b]" is not normally possible in
% LaTeX2e). It also provides a command:
%\fnbelowfloat
% to enable the placement of footnotes below bottom floats (the standard
% LaTeX2e kernel puts them above bottom floats). This is an invasive package
% which rewrites many portions of the LaTeX2e float routines. It may not work
% with other packages that modify the LaTeX2e float routines. The latest
% version and documentation can be obtained at:
% http://www.ctan.org/tex-archive/macros/latex/contrib/sttools/
% Documentation is contained in the stfloats.sty comments as well as in the
% presfull.pdf file. Do not use the stfloats baselinefloat ability as IEEE
% does not allow \baselineskip to stretch. Authors submitting work to the
% IEEE should note that IEEE rarely uses double column equations and
% that authors should try to avoid such use. Do not be tempted to use the
% cuted.sty or midfloat.sty packages (also by Sigitas Tolusis) as IEEE does
% not format its papers in such ways.





% *** PDF, URL AND HYPERLINK PACKAGES ***
%
%\usepackage{url}
% url.sty was written by Donald Arseneau. It provides better support for
% handling and breaking URLs. url.sty is already installed on most LaTeX
% systems. The latest version can be obtained at:
% http://www.ctan.org/tex-archive/macros/latex/contrib/misc/
% Read the url.sty source comments for usage information. Basically,
% \url{my_url_here}.





% *** Do not adjust lengths that control margins, column widths, etc. ***
% *** Do not use packages that alter fonts (such as pslatex).         ***
% There should be no need to do such things with IEEEtran.cls V1.6 and later.
% (Unless specifically asked to do so by the journal or conference you plan
% to submit to, of course. )


% correct bad hyphenation here
\hyphenation{op-tical net-works semi-conduc-tor}

\usepackage{color}
\def\func#1#2{\mbox{\textrm{\bf{\sc{#1}}}}\ensuremath{(~{#2}~)}}
\def\array#1#2{\mbox{\textrm{\bf{\sc{#1}}}}\ensuremath{[{#2}]}}
\def\arraysize#1{\ensuremath{\size{\mbox{\textrm{\bf{\sc{#1}}}}}}}
\def\var#1{\mbox{\textrm{\bf{\sc{#1}}}}}
\definecolor{gray}{rgb}{0.3,0.3,0.3}

\newtheorem{fact}{Fact}[section]
\newtheorem{lemma}[fact]{Lemma}
\newtheorem{theorem}[fact]{Theorem}
\newtheorem{definition}[fact]{Definition}
\newtheorem{corollary}[fact]{Corollary}
\newtheorem{proposition}[fact]{Proposition}
\newtheorem{claim}[fact]{Claim}
\newtheorem{exercise}[fact]{Exercise}

% math notations
\newcommand{\size}[1]{\ensuremath{\left|#1\right|}}
\newcommand{\ceil}[1]{\ensuremath{\left\lceil#1\right\rceil}}
\newcommand{\floor}[1]{\ensuremath{\left\lfloor#1\right\rfloor}}

% asymptotic notations
\newcommand{\Oh}[1]{\ensuremath{{\mathcal O}\left({#1}\right)}}
\newcommand{\LOh}[1]{\ensuremath{{\mathcal O}\left({#1}\right.}}
\newcommand{\ROh}[1]{\ensuremath{{\mathcal O}\left.{#1}\right)}}
\newcommand{\oh}[1]{\ensuremath{{o}\left({#1}\right)}}
\newcommand{\Om}[1]{\ensuremath{{\Omega}\left({#1}\right)}}
\newcommand{\om}[1]{\ensuremath{{\omega}\left({#1}\right)}}
\newcommand{\Th}[1]{\ensuremath{{\Theta}\left({#1}\right)}}

% pseudocode notations
\newcommand{\xinvariant}{\mbox{\bf{\em{invariant~}}}}
\newcommand{\xif}{\mbox{\bf{\em{if~}}}}
\newcommand{\xthen}{\mbox{\bf{\em{then~}}}}
\newcommand{\xelse}{\mbox{\bf{\em{else~}}}}
\newcommand{\xelseif}{\mbox{\bf{\em{elif~}}}}
\newcommand{\xfi}{\mbox{\bf{\em{fi~}}}}
\newcommand{\xendif}{\mbox{\bf{\em{endif~}}}}
\newcommand{\xcase}{\mbox{\bf{\em{case~}}}}
\newcommand{\xendcase}{\mbox{\bf{\em{endcase~}}}}
\newcommand{\xbreak}{\mbox{\bf{\em{break~}}}}
\newcommand{\xfor}{\mbox{\bf{\em{for~}}}}
\newcommand{\xin}{\mbox{\bf{\em{in~}}}}
\newcommand{\xto}{\mbox{\bf{\em{to~}}}}
\newcommand{\xby}{\mbox{\bf{\em{by~}}}}
\newcommand{\xdownto}{\mbox{\bf{\em{downto~}}}}
\newcommand{\xdo}{\mbox{\bf{\em{do~}}}}
\newcommand{\xrof}{\mbox{\bf{\em{rof~}}}}
\newcommand{\xwhile}{\mbox{\bf{\em{while~}}}}
\newcommand{\xendwhile}{\mbox{\bf{\em{endwhile~}}}}
\newcommand{\xand}{\mbox{\bf{\em{and~}}}}
\newcommand{\xor}{\mbox{\bf{\em{or~}}}}
\newcommand{\xerror}{\mbox{\bf{\em{error~}}}}
\newcommand{\xreturn}{\mbox{\bf{\em{return~}}}}
\newcommand{\xparallel}{\mbox{\bf{\em{parallel~}}}}
\newcommand{\xspawn}{\mbox{\bf{\em{spawn~}}}}
\newcommand{\xsync}{\mbox{\bf{\em{sync~}}}}
\newcommand{\xarray}{\mbox{\bf{\em{array}}}}
\newcommand{\xqueue}{\mbox{\bf{\em{queue}}}}
\newcommand{\xfalse}{\mbox{\bf{\em{false}}}}
\newcommand{\xtrue}{\mbox{\bf{\em{true}}}}
\newcommand{\T}{\mbox{\hspace{0.5cm}}}
\newcommand{\m}[1]{\ensuremath{\mathcal{#1}}}



\begin{document}
%
% paper title
% can use linebreaks \\ within to get better formatting as desired
\title{Parallel Breadth First Search}


% author names and affiliations
% use a multiple column layout for up to three different
% affiliations
\author{
    %\IEEEauthorblockN{Rezaul Chowdhury}
    %\and
    \IEEEauthorblockN{Yonatan Fogel}
    \IEEEauthorblockA{Computer Science\\
        Stony Brook University\\
    Stony Brook, New York, 11744}
    %\and
    %\IEEEauthorblockN{Jesmin Jahan}
    %\IEEEauthorblockA{Twentieth Century Fox\\
    %Springfield, USA\\
    %Email: homer@thesimpsons.com}
    %\and
    %\IEEEauthorblockN{James Kirk\\ and Montgomery Scott}
    %\IEEEauthorblockA{Starfleet Academy\\
    %San Francisco, California 96678-2391\\
    %Telephone: (800) 555--1212\\
    %Fax: (888) 555--1212}
}

% conference papers do not typically use \thanks and this command
% is locked out in conference mode. If really needed, such as for
% the acknowledgment of grants, issue a \IEEEoverridecommandlockouts
% after \documentclass

% for over three affiliations, or if they all won't fit within the width
% of the page, use this alternative format:
% 
%\author{\IEEEauthorblockN{Michael Shell\IEEEauthorrefmark{1},
%Homer Simpson\IEEEauthorrefmark{2},
%James Kirk\IEEEauthorrefmark{3}, 
%Montgomery Scott\IEEEauthorrefmark{3} and
%Eldon Tyrell\IEEEauthorrefmark{4}}
%\IEEEauthorblockA{\IEEEauthorrefmark{1}School of Electrical and Computer Engineering\\
%Georgia Institute of Technology,
%Atlanta, Georgia 30332--0250\\ Email: see http://www.michaelshell.org/contact.html}
%\IEEEauthorblockA{\IEEEauthorrefmark{2}Twentieth Century Fox, Springfield, USA\\
%Email: homer@thesimpsons.com}
%\IEEEauthorblockA{\IEEEauthorrefmark{3}Starfleet Academy, San Francisco, California 96678-2391\\
%Telephone: (800) 555--1212, Fax: (888) 555--1212}
%\IEEEauthorblockA{\IEEEauthorrefmark{4}Tyrell Inc., 123 Replicant Street, Los Angeles, California 90210--4321}}




% use for special paper notices
%\IEEEspecialpapernotice{(Invited Paper)}




% make the title area
\maketitle


% IEEEtran.cls defaults to using nonbold math in the Abstract.
% This preserves the distinction between vectors and scalars. However,
% if the conference you are submitting to favors bold math in the abstract,
% then you can use LaTeX's standard command \boldmath at the very start
% of the abstract to achieve this. Many IEEE journals/conferences frown on
% math in the abstract anyway.

\begin{abstract}
%\boldmath
    This is an $n-1$ level abstract
\end{abstract}


% no keywords




% For peer review papers, you can put extra information on the cover
% page as needed:
% \ifCLASSOPTIONpeerreview
% \begin{center} \bfseries EDICS Category: 3-BBND \end{center}
% \fi
%
% For peerreview papers, this IEEEtran command inserts a page break and
% creates the second title. It will be ignored for other modes.
\IEEEpeerreviewmaketitle

\section{Introduction}
% no \IEEEPARstart

\iffalse
"we need fast parallel BFS, but dividing the work takes too much time and is really complicated. also we cannot easily schedule work to reduce message passing."
"this paper discusses a new technique to effiiciently and determinstically divide the work to be done for BFS across multiple processing elements."
"this work is evaluated using the MODEL model to accurately but succinctly characterize modern parallel processing architectures [CITE,CITE]"
"we show our technique is optimal under these conditions: (1) one, (2) two, and (3) three."
\fi
\subsection{Subsection Heading Here}
Subsection text here.

\subsubsection{Subsubsection Heading Here}
Subsubsection text here.


\section{Related Work}
% no \IEEEPARstart

\subsection{Terms}
\begin{itemize}
    \item $G=(V,E)$ is the graph.
    \item $V$ is the set of nodes.
    \item $n=\size{V}$.
    \item $m=\size{E}$.
    \item $\Gamma(u)$ is the set of nodes adjacent to node $u$.
    \item $T_s$ is the running time for the most efficient serial algorithm.
    \item $T_1 \equiv \var{Work}$ is the running time for a parallel algorithm running on one processing elements.
    \item $T_P$ is the running time for a parallel algorithm running on $P$ processing elements.
    \item $T_\infty \equiv \var{Span}$ is the critical path for a parallel algorithm or the time it takes given infinite processing elements.
    \item $W_P \leq pT_P$ is the total amount of work for $P$ processing elements.  Idle processing elements do not count towards $W_P$.
\end{itemize}


We consider only \defn{level-synchronous} BFS algorithms, which find all nodes at distance $0\leq d \leq \size{V}$ before any nodes at distance $d^\prime > d$.

The standard serial approach for level-synchronous BFS can be seen in Fig. \ref{fi:general-bfs}.
\codefigure{
    $\func{Serial-BFS}{V, ~\Gamma, ~s}$

    \begin{enumerate}
        \item \xfor each vertex $u \in V$
        \item \T $\array{Dist}{u} \gets \infty$
        \item $\array{Dist}{s} \gets 0$
        \item $\func{Enqueue}{Q_{in}, ~v}$
        \item \xwhile $Q_{in} \neq \emptyset$ \xdo
        \item \T $Q_{out} \gets \emptyset$
        \item \T \xwhile $Q_{in} \neq \emptyset$ \xdo
        \item \T \T $u \gets \func{Dequeue}{Q_{in}}$
        \item \T \T \xfor each vertex $v \in \Gamma(u)$ \xdo
        \item \T \T \T \xif $\array{Dist}{v} = \infty$ \xthen \label{bfs-race-1}
        \item \T \T \T \T $\array{Dist}{v} \gets \array{Dist}{u} + 1$
        \item \T \T \T \T $\func{Enqueue}{Q_{out}, ~v}$
        \item \T $Q_{in} \gets Q_{out}$\label{bfs-race-2}
    \end{enumerate}
}{Standard serial level-synchronous breadth-first search implementation on a graph $G=(V,E)$ with source vertex $s\in V$. $\Gamma(u)$ is the adjacency list for node $u$.}{general-bfs}

The bottlenecks for parallelization are the FIFO queue and the \var{Dist} array.  The FIFO is fast but is inherently serial.  When running in parallel, there is a benign race condition on lines \ref{bfs-race-1}--\ref{bfs-race-2} of Fig. \ref{fi:general-bfs}.  Multiple threads can enqueue the same vertex.  The race is benign and rare: it can create extra work but does not affect correctness.  This race is sometimes dealt with mutual exclusion, atomic instructions, or by simply performing the extra work when the race occurs.

\subsection*{Existing algorithms}
\defn{MIT-Bag}\cite{mit-bag} uses reducer hyperobjects and a new \defn{bag} data structure to replace the FIFO.
This algorithm relies on a work-stealing scheduler and runs in $T_P = \Oh{n/P+m/P+D\lg^3(n/D)}$ w.h.p in $n$.
MIT-Bag uses mutual exclusion to deal with the race conditions on \var{Dist} for analysis, but in practice recommends simply doing the extra work.

\defn{CRAY-BFS}\cite{cray-bfs} uses special hardware available on the Cray MTA-2 platform.  It writes to the FIFO in parallel by using atomic increments, and uses special hardware mutexes for every $64$-bit word.

\defn{Block-BFS}\cite{block-queue-bfs} uses a block-accessed queue to replace the FIFO.  It uses atomic primitives to allocate blocks from the FIFO that are small, but not so small so that they do not use atomics too often.

\defn{Distinguished-BFS}\cite{distinguished-bfs} is a non level-synchronous BFS algorithm for arbitrary sparse graphs. Distinguished-BFS runs in $T_P = \Oh{n^\epsilon}$ time with $\Oh{mn^{1-2\epsilon}}$ processors, if $m\geq n^{2-3\epsilon}$.  The basic idea is to contract the graph to distinguished and then superdistinguished vertexes, at which point the graph will be dense.

%TODO: search for prefix sum bfs (GPU), search Guy Blelloch, search recent BFS


\section{Parallel Prefix Sum}
% no \IEEEPARstart

Parallel prefix sum is a common primitive used in many parallel algorithms.  In the CILK model, the standard (\Oh{\log n} parallel fors) parallel prefix sum  on $n$ elements runs in $T_\infty=\Oh{\log^2 n}$ or $T_P=\Oh{n/P + \log ^2 P}$.  In the PRAM model it runs in $T_\infty = \Oh{\log n}$.

We introduce a divide-and-conquer version of parallel prefix sum that runs in $T_P = \Oh{n/P + \log P}$. 

\codefigure{
    \medskip\noindent\func{Parallel-Prefix-Sum}{V, ~\var{Grain-Size}}
    %\vspace{0.1cm}

    \noindent
    % TODO: comment?
    {\color{gray} ${V}[0:n-1]$ is a sequence of $n$ integers.  This function replaces $V[i]$ with $\sum_{0\leq j \leq i}{V[j]}$}
    \noindent
    \begin{enumerate}
        \item \xif $\size{V} > 1$ \xthen
        \item \T \func{Parallel-Prefix-Sum-Up}{V, ~\var{Grain-Size}, ~0, ~n}
        \item \T \func{Parallel-Prefix-Sum-Down}{V, ~\var{Grain-Size}, ~0, ~n, ~\xfalse, ~0}
    \end{enumerate}
}{}{parallel-prefix-sum}

\codefigure{
    \func{Parallel-Prefix-Sum-Up}{V, ~\var{Grain-Size}, ~start, ~limit}
    \noindent
    % TODO: comment?
    \noindent

    \begin{enumerate}
        \item $size \gets limit - start$
        \item \xif $size \leq \var{Grain-Size}$ \xthen
        \item \T \xreturn $\func{Serial-Prefix-Sum}{V, ~start, ~limit}$
        \item \xelse
        \item \T $mid \gets \floor{\frac{start+limit}{2}}$
        \item \T $x \gets \xspawn \func{Parallel-Prefix-Sum-Up}{V, ~\var{Grain-Size}, ~start, ~mid}$
        \item \T $y \gets \func{Parallel-Prefix-Sum-Up}{V, ~\var{Grain-Size}, ~mid, ~limit}$
        \item \T \xsync
        \item \T $V[limit-1] \gets x+y$
        \item \T \xreturn $x+y$
    \end{enumerate}
}{}{parallel-prefix-sum-up}

\codefigure{
    \medskip\noindent\func{Parallel-Prefix-Sum-Down}{V, ~start, ~limit, ~rightmost\_excluded, ~partial\_sum}
    \noindent
    % TODO: comment?
    \noindent

    \begin{enumerate}
        \item $size \gets limit - start$
        \item \xif $size \leq \var{Grain-Size}$ \xthen
        \item \T $\func{Serial-Prefix-Sum-Down}{V, ~start, ~limit, ~partial\_sum, ~rightmost\_excluded}$
        \item \T \xreturn
        \item \xelse
        \item \T $mid \gets \floor{\frac{start+limit}{2}}$
        \item \T $sum\_left \gets V[mid-1]$
        \item \T \xspawn $\func{Parallel-Prefix-Sum-Down}{V, ~\var{Grain-Size}, ~start, ~mid, ~\xfalse, ~partial\_sum}$
        \item \T \xif $\neg rightmost\_excluded$ \xthen
        \item \T \T $V[limit-1] \gets V[limit-1] + partial\_sum$
        \item \T \xif $limit - mid > 1$
        \item \T \T $\func{Parallel-Prefix-Sum-Down}{V, ~\var{Grain-Size}, ~mid, ~limit, ~\xtrue, ~partial\_sum + sum\_left}$
    \end{enumerate}
}{}{parallel-prefix-sum-down}

\codefigure{
    \func{Serial-Prefix-Sum}{V, ~start, ~limit}
    \noindent
    % TODO: comment?
    \noindent

    \begin{enumerate}
        \item $V[start] \gets V[start] + partial\_sum$
        \item \xfor $i \gets start+1$ \xto $limit-1$ \xdo
        \item \T $V[i] \gets V[i] + V[i-1]$
        \item \xreturn $V[limit-1]$
    \end{enumerate}
}{}{serial-prefix-sum}

\codefigure{
    \medskip\noindent\func{Serial-Prefix-Sum-Down}{V, ~start, ~limit, ~rightmost\_excluded, ~partial\_sum}
    \noindent
    % TODO: comment?
    \noindent

    \begin{enumerate}
        \item \xfor $i \gets start$ \xto $limit-2$ \xdo
        \item \T $V[start] \gets V[start] + partial\_sum$
        \item \xif $\neg rightmost\_excluded$ \xthen
        \item \T $V[limit-1] \gets V[limit-1] + partial\_sum$
    \end{enumerate}
}{}{serial-prefix-sum-down}



\section{FBFS}

\subsection{Lower Bound}
We establish a lower bound for the parallel running time of level-synchronous BFS.
Let \defn{D} be the diameter of a graph $G=(V,E)$, $n=\size{V}, m=\size{E}$.

For level-synchronous BFS, $T_s=\Oh{n+m+D}=\Oh{n+m}$ is a lower bound for both $T_1$ and $W_p$. 
Let $n_l$ be the number of nodes at distance $l$ from the source vertex, and $m_l$ be the sum of out-degrees
of the $n_l$ nodes at distance $l$.
To establish the lower bound for $T_p$, consider a graph where for every level $l$, $(n_l+m_l) = \Oh(p)$.
$n_l+m_l$ 

BFS algo
\begin{verbatim}
parallel for u gets 0 .. n - 1
\end{verbatim}

\begin{verbatim}
parallel for i = 0 to p-1
if i = 0
var{offset} = 0
else
var{offset} = D[\frac{iN}{p}
    var{offset}


    \end{verbatim}

\codefigure{
    $\func{Parallel-BFS}{V, ~\Gamma, ~\gamma, ~s, ~p_{max}}$

    {\color{gray} ${V}[0:n-1]$ are the $n$ nodes in the graph.  $\Gamma[u]$ is the sequence of adjacent nodes to node $u$. $\gamma[u] = \size{\Gamma[u]}$.  $s$ is the source vertex from which distance is calculated.  $p_{max}$ is the maximum number of processors to use.  Returns $\array{Dist}{0:n-1}$ which represents the distance from $s$ to each vertex.}
    \noindent
                        % TODO: comment?
    \noindent

    \begin{enumerate}
        \item \xparallel \xfor $u \gets 0$ \xto $n-1$ \xdo
        \item \T $\array{Dist}{u} \gets \infty$
        \item \T $\array{Owner}{u} \gets \infty$
        \item $\array{Dist}{s} \gets 0$
        \item $\array{Owner}{s} \gets 0$
        \item \xif $\gamma[s] = 0$ \xthen
        \item \T $\xreturn \var{Dist}$
        \item $\var{Input} \gets \xarray[0:0]$
        \item $\array{Input}{0} \gets s$
        \item $\var{Level} \gets 0$
        \item $p \gets 1$
        \item \xwhile $\arraysize{Input} \neq 0$ \xdo
        \item \T $\var{Level} \gets \var{Level} + 1$
        \item \T $\var{N} \gets \arraysize{Input}$
        \item \T $\var{Work} \gets \xarray[0:N-1]$
        \item \T \xparallel \xfor $u \gets 0$ \xto $N-1$ \xdo
        \item \T \T $\array{Work}{u} \gets \gamma[\array{Input}{u}]$
            \comment{}
        \item \T $\func{Parallel-Prefix-Sum}{\var{Work}, \floor{\frac{N}{p}}}$
        \item \T $W \gets \array{Work}{N-1}$
        \item \T $p \gets \func{Min}{p_{max}, ~W}$

        \item \T $\var{Sublist} \gets \func{Find-Sublist}{\var{Work}, ~W, ~p}$

        \item \T $\var{Q} \gets \func{Level-To-Queues}{\var{Input}, ~\var{Work}, ~\var{Sublist}, \var{Dist}, ~\Gamma, ~\gamma, ~W, ~p, ~\var{Level}}$
                                %TODO call here
        \item \T $\var{Sizes} \gets \xarray[0:p-1]$
        \item \T \xparallel \xfor $i \gets 0$ \xto $p-1$ \xdo
        \item \T \T $\var{Q-New} \gets \xqueue$
        \item \T \T \T \xfor $v$ \xin $\array{Q}{i}$ \xdo
        \item \T \T \T \T \xif $\array{Owner}{v} = i$ \xthen
        \item \T \T \T \T \T $\var{Q-New}.\func{Enqueue}{v}$
        \item \T \T $\array{Q}{i} \gets \var{Q-New}$
        \item \T \T $\array{Sizes}{i} \gets \size{\array{Q}{i}}$
        \item \T $\func{Parallel-Prefix-Sum}{\var{Sizes}, 1}$
        \item \T $\var{Input} \gets \xarray[0:\array{Sizes}{p-1}]$
        \item \T \xparallel \xfor $i \gets 0$ \xto $p-1$ \xdo
        \item \T \T \xif $i = 0$ \xthen
        \item \T \T \T $\var{Offset} \gets 0$
        \item \T \T \xelse
        \item \T \T \T $\var{Offset} \gets \array{Sizes}{i-1}$
        \item \T \T \xfor $j \gets \var{Offset}$ \xto $\var{Offset} + \size{\array{Q}{i}}$ \xdo
        \item \T \T \T $\array{Input}{\var{Offset}} \gets \array{Q}{i}.\func{Dequeue}{}$
    \end{enumerate}
}{}{parallel-bfs}

\codefigure{
    $\func{Find-Sublist}{\var{Work}, ~W, ~p}$

                        % TODO: comment?
    \noindent

    \begin{enumerate}
        \item $N \gets \arraysize{Work}$
        \item $p_n \gets \func{Min}{p, ~N}$
        \item $\var{Sublist} \gets \xarray[0:p-1]$
        \item $\var{RangesStart} \gets \xarray[0:p_n-1]$
        \item $\var{RangesEnd} \gets \xarray[0:p_n-1]$
        \item \xparallel \xfor $i \gets 0$ \xto $p_n-1$ \xdo
        \item \T \xif $i = 0$ \xthen
        \item \T \T $\var{FirstDegree} \gets 0$
        \item \T \xelse
        \item \T \T $\var{FirstDegree} \gets \array{Work}{\floor{\frac{i N}{p_n}} - 1}$
        \item \T $\var{FirstDegreeNext} \gets \array{Work}{\floor{\frac{(i+1) N}{p_n}}-1}$
        \item \T $\array{RangesStart}{i} \gets \ceil{\frac{p \cdot \var{FirstDegree}}{W}}$
        \item \T $\array{RangesEnd}{i} \gets \ceil{\frac{p \cdot \var{FirstDegreeNext}}{W}}-1$
        \item \xparallel \xfor $i \gets 0$ \xto $p_n-1$ \xdo
        \item \T \xif $\array{RangesStart}{i} \leq \array{RangesEnd}{i}$ \xthen
        \item \T \T \xparallel \xfor $j \gets \array{RangesStart}{i}$ \xto $\array{RangesEnd}{i}$ \xdo
            \comment{Use cores $\array{RangesStart}{i}$ to $\array{RangesEnd}{i}$}
        \item \T \T \T $\array{Sublist}{j} \gets i$
        \item \xreturn \var{Sublist}
    \end{enumerate}
}{}{find-sublist}

\codefigure{
    $\func{Level-To-Queues}{\var{Input}, ~\var{Work}, ~\var{Sublist}, \var{Dist}, ~\Gamma, ~\gamma, ~W, ~p, ~\var{Level}}$

                        % TODO: comment?
    \noindent

    \begin{enumerate}
        \item $N \gets \arraysize{Work}$
        \item $p_n \gets \func{Min}{p, ~N}$
        \item $\var{Q} \gets \xarray[0:p-1]$
        \item \xparallel \xfor $i \gets 0$ \xto $p-1$ \xdo
        \item \T $\array{Q}{i}.\func{Clear}{}$
        \item \T $\var{FirstDegree} \gets \floor{\frac{iW}{p}}$
        \item \T $\var{WorkItems} \gets \floor{\frac{(i+1)W}{p}} - \var{FirstDegree}$
        \item \T $\var{Vertex} \gets \func{Binary-Search-For-Index}{\var{Work}, \floor{\frac{N\cdot\array{Sublist}{i}}{p_n}}, \floor{\frac{N\cdot(\array{Sublist}{i}+1)}{p_n}}, \var{FirstDegree}+1}$
        \item \T $\var{Degree} \gets \var{FirstDegree} - \array{Work}{\var{Vertex}-1}$
        \item \T \xwhile $\var{WorkItems} > 0$ \xdo
        \item \T \T $u \gets \array{Input}{\var{Vertex}}$
        \item \T \T $\var{Limit} \gets \func{Min}{\var{WorkItems} + \var{Degree}, ~\gamma[u]}$
        \item \T \T \xfor $j \gets \var{Degree}$ \xto $\var{Limit}$ \xdo
        \item \T \T \T $v \gets \Gamma[u][j]$
        \item \T \T \T \xif $\array{Dist}{v} = \infty$ \xthen
        \item \T \T \T \T $\array{Dist}{v} \gets \var{Level}$
            \comment{Benign race condition.  All threads write the same value.}
        \item \T \T \T \T $\array{Owner}{v} \gets i$
            \comment{Benign race condition.  One thread's value will win.}
        \item \T \T \T \T \xif $\gamma[v] > 0$ \xthen
        \item \T \T \T \T \T $\array{Q}{i}.\func{Enqueue}{v}$
        \item \T \T $\var{WorkItems} \gets \var{WorkItems} - \var{Degree}$
        \item \T \T $\var{Degree} \gets 0$
        \item \T \T $\var{Vertex} \gets \var{Vertex}+1$
        \item \xreturn \var{Q}
    \end{enumerate}
}{}{level-to-queues}
% no \IEEEPARstart




\section{Analysis}
% no \IEEEPARstart


\section{Conclusion}



\section{Future Work}


% conference papers do not normally have an appendix

% use section* for acknowledgement
%\section*{Acknowledgment}
%The authors would like to thank...

%Rezaul Chowdhury
%Jesmin Jahan
%Michael Bender


\bibliographystyle{IEEEtran}
\bibliography{IEEEabrv,bib/paper}









% trigger a \newpage just before the given reference
% number - used to balance the columns on the last page
% adjust value as needed - may need to be readjusted if
% the document is modified later
%\IEEEtriggeratref{8}
% The "triggered" command can be changed if desired:
%\IEEEtriggercmd{\enlargethispage{-5in}}

% references section

% can use a bibliography generated by BibTeX as a .bbl file
% BibTeX documentation can be easily obtained at:
% http://www.ctan.org/tex-archive/biblio/bibtex/contrib/doc/
% The IEEEtran BibTeX style support page is at:
% http://www.michaelshell.org/tex/ieeetran/bibtex/
%\bibliographystyle{IEEEtran}
% argument is your BibTeX string definitions and bibliography database(s)
%\bibliography{IEEEabrv,../bib/paper}
%
% <OR> manually copy in the resultant .bbl file
% set second argument of \begin to the number of references
% (used to reserve space for the reference number labels box)
%\begin{thebibliography}{1}
%
%\bibitem{IEEEhowto:kopka}
%H.~Kopka and P.~W. Daly, \emph{A Guide to \LaTeX}, 3rd~ed.\hskip 1em plus
%  0.5em minus 0.4em\relax Harlow, England: Addison-Wesley, 1999.
%
%\end{thebibliography}




% that's all folks
\end{document}


